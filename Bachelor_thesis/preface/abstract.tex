The concept of a thinking machine has always teased the human mind. In the last years, with the development of Machine Learning (ML) and Neural Networks (NN), the topic has become something more than a simple suggestion. Certainly, the  prediction of the possible applications of such technologies have helped to increase the enthusiasm towards it. In this context it is possible to mention the possibility of discovering new drugs or automating certain tasks.
Research as well as private companies have started to invest an increasing amount of money and time in order to develop the theoretical and applicative aspects of ML. This has leaded to the launch of many applications which deal with computer vision, speech as well as objects recognition, etc. In recent times, applications has been found also in physics, economy, bionformatics and marketing.


We implement the concept of ML with NN to non-interacting bosons in an elliptical harmonic trap, a system used to simulate the behaviour of a Bose-Einstein condensate. The energy is computed through Variational Monte Carlo methods, where the trial wave-function is represented by the Gaussian-Binary Restricted Boltzmann Machine (RBM). Two sampling methods are discussed: Brute-Force Metropolis and Importance Sampling Metropolis-Hastings. The variational parameters are updated with the Stochastic Gradient Descent method. The errors are computed through the Blocking method which takes into account correlations between data. An outline of the algorithms as well as of the general structure of the code is provided. 

%In the process of developing the code, we benchmarked it in two cases. 
To benchmark our code, we consider two cases: non-interacting bosons in a spherical harmonic trap and non-interacting bosons in an elliptical (along the z direction) harmonic trap. In the first one, the results agree with the previsions with good precision. In the second case, the results are not compatible with the analytical ones: the RBM seem to be ineffective in learning the elliptical symmetry. The problem is fixed by biasing the RBM's parameters which involve the z-ax. Then, we apply the code to the case of interacting (through hard-core potential) bosons in an elliptical harmonic trap. The data obtained are compared to the results furnished by the Gross-Pitaevskii (GP) equation in \cite{DalfString}. Even though, in some cases, our data underestimate the GP's one, the overall trend seem to be compatible. The final result $E_L/N_p=\SI{2.63\pm 0.02}{}\ \hbar \omega_{ho}$ with $N_p=100$ is found to agree well with the GP prediction.
 
 

%0.05 learning rates for NH>=4 in elliptical